{\bfseries{Requisitos\+:}}

Se debe tener instalado para el correcto funcionamiento\+:

-\/Maven 3.\+6.\+3

-\/JUnit 4

-\/Open\+JDK 15

-\/XAMPP (Para iniciar la parte servidora)

{\bfseries{Ejecución de test\+:}}

Para ejecutar los test unitarios se debe usar el comando\+: {\ttfamily mvn clean test}

Con ello dentro del directorio \char`\"{}target/site/jacoco\char`\"{} esta \char`\"{}index.\+html\char`\"{} que te muestra el coverage del proyecto. Además, dentro de \char`\"{}target/contiperf-\/report\char`\"{} existe otro \char`\"{}index.\+html\char`\"{} para mostrar los test de rendimiento y sus resultados.

Para la ejecucion de la parte de integración se debe usar el comando\+: {\ttfamily mvn verify -\/Pintegracion}

{\bfseries{Configuración previa\+:}}

Para inicializar el servidor My\+SQL debemos introducir las siguientes sentencias dentro del administrador de XAMPP en la parte de SQL\+: 
\begin{DoxyCode}{0}
\DoxyCodeLine{DROP SCHEMA IF EXISTS gamedb;}
\DoxyCodeLine{DROP USER IF EXISTS 'rpg'@'localhost';}
\DoxyCodeLine{}
\DoxyCodeLine{CREATE SCHEMA gamedb;}
\DoxyCodeLine{CREATE USER IF NOT EXISTS 'rpg'@'localhost' IDENTIFIED BY 'rpg';}
\DoxyCodeLine{}
\DoxyCodeLine{GRANT ALL ON gamedb.* TO 'rpg'@'localhost';}

\end{DoxyCode}


Tambien se debe arrancar la parte de Apache pulsando el boton \char`\"{}\+Start\char`\"{} al igual que en My\+SQL.

{\bfseries{Configuración\+:}}

Para la correcta inicialización del juego se deben seguir los siguientes pasos en orden y un por uno\+:

{\bfseries{1.}} Descargar el zip del proyecto.

{\bfseries{2.}} Descomprimir el .zip en la carpeta que se desee.

{\bfseries{3.}} Abrir la carpeta descomprimida (Juego\+Rpg-\/main).

{\bfseries{4.}} Seleccionar la carpeta RPG.

{\bfseries{5.}} Estando en la carpeta RPG, escribir en la barra del buscador de archivos {\ttfamily cmd}.

{\bfseries{6.}} A continuación se abrira la consola de comandos, escriba en orden estos comandos\+: 
\begin{DoxyCode}{0}
\DoxyCodeLine{mvn clean install }
\DoxyCodeLine{mvn datanucleus:enhance }
\DoxyCodeLine{mvn datanucleus:schema-\/create }

\end{DoxyCode}


{\bfseries{7.}} Si todo funciona bien, deberia de terminar la compilación sin mostrar ningun error.

{\bfseries{8.}} Una vez acabado este paso introduzca el siguiente comando {\ttfamily mvn exec\+:java}.

{\bfseries{9.}} Ahora cargara la ventana principal del juego y podra jugar sin problemas.

{\bfseries{10.}} Si quiere limpiar la base de datos use este comando {\ttfamily mvn datanucleus\+:schema-\/delete}. 